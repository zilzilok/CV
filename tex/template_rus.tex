%%%%%%%%%%%%%%%%%%%%%%%%%%%%%%%%%%%%%%%%%
% Medium Length Professional CV
% LaTeX Template
% Version 3.0 (December 17, 2022)
%
% This template originates from:
% https://www.LaTeXTemplates.com
%
% Author:
% Vel (vel@latextemplates.com)
%
% Original author:
% Trey Hunner (http://www.treyhunner.com/)
%
% License:
% CC BY-NC-SA 4.0 (https://creativecommons.org/licenses/by-nc-sa/4.0/)
%
%%%%%%%%%%%%%%%%%%%%%%%%%%%%%%%%%%%%%%%%%

%----------------------------------------------------------------------------------------
%	PACKAGES AND OTHER DOCUMENT CONFIGURATIONS
%----------------------------------------------------------------------------------------

\documentclass[
	%a4paper, % Uncomment for A4 paper size (default is US letter)
	11pt, % Default font size, can use 10pt, 11pt or 12pt
]{resume} % Use the resume class

\usepackage{ebgaramond} % Use the EB Garamond font

% Для кириллицы
\usepackage[T2A,T1]{fontenc}
\usepackage[utf8]{inputenc}
\usepackage[russian]{babel}
\usepackage{tempora}

%------------------------------------------------

\name{Тагир Сафин} % Your name to appear at the top

% You can use the \address command up to 3 times for 3 different addresses or pieces of contact information
% Any new lines (\\) you use in the \address commands will be converted to symbols, so each address will appear as a single line.

\address{\\ safintagirn@gmail.com} % Contact information
\address{\\ https://github.com/zilzilok}

%----------------------------------------------------------------------------------------

\begin{document}

%----------------------------------------------------------------------------------------
%	EDUCATION SECTION
%----------------------------------------------------------------------------------------

\begin{rSection}{Образование}
	\textbf{Высшая Школа Экономики, Факультет Компьютерных Наук} \hfill {Сентябрь 2018 - Июль 2022} \\
	\textit{Бакалавр, Программная Инженерия} \hfill \textit{Москва}

	\textbf{Дипломная работа:} Приложение по автоматизации бизнес процессов с применением технологий BPM
	
	\textbf{GPA:} 7.57 из 10
	
\end{rSection}

%----------------------------------------------------------------------------------------
%	WORK EXPERIENCE SECTION
%----------------------------------------------------------------------------------------

\begin{rSection}{Опыт Работы}

	\begin{rSubsection}{Яндекс}{Апрель 2021 - наст. время}{Разработчик программного обеспечения}{Москва, Белград}
		\item Разрабатывал и реализовывал функционал в интересах группы качества Яндекс Маркета.
		\item Поддерживал и улучшал различные микросервисы
		\item Использовал в работе различные внутренние технологии, утилиты: монорепозиторий и его VCS, система сборки, YT (аналог Hadoop), LogBroker (аналог Apache Kafka), Yandex Object Storage и др.
	\end{rSubsection}

\end{rSection}

%----------------------------------------------------------------------------------------
%	TECHNICAL STRENGTHS SECTION
%----------------------------------------------------------------------------------------

\begin{rSection}{Технические Навыки}

	\begin{tabular}{@{} >{\bfseries}l @{\hspace{6ex}} l @{}}
		Основные навыки & Java, Kotlin, Spring Framework \\
		Понимание в структурировании API & REST, OpenApi (и его составляющие) \\
		СУБД  &  PostgreSQL, ClickHouse \\
		Из необычного  & Camunda Platform, Yandex Cloud SpeechKit
	\end{tabular}

\end{rSection}

%----------------------------------------------------------------------------------------
%	HOBBY SECTION
%----------------------------------------------------------------------------------------

\begin{rSection}{Хобби и Интересы}

	Музыка, барабаны.

\end{rSection}

%----------------------------------------------------------------------------------------

\end{document}
